\section{Single\hyp{}Threaded Applications}

We use a simple vector addition example to illustrate the fundamentals of GMAC programming. In this 
section, our goal is to build an application that reads two variable\hyp{}sized input floating point 
vectors from a file, computes a third output floating point vector as an addition of the two input 
vectors, and stores that vector in a file.

\subsection{The Kernel code}
Listing~\ref{lst:hpe:vecadd} shows the vector addition kernel code. This code might be stored as a 
global constant char array (\ie \texttt{const char *}) in the application source code, or as a text 
file. We will assume the latter option, where the code in Listing~\ref{lst:hpe:vecadd} is in a file 
called \texttt{vecadd.cl} in the same directory than the application binary.

\lstinputlisting[float,
    language=C++,
    frame=tb,
    caption={Vector addition kernel code.},
    label={lst:hpe:vecadd}]{hpe/vecadd.cl}

A minor comment about the code in Listing~\ref{lst:hpe:vecadd} is the usage of \texttt{\_\_global 
float const * restrict} for the input  vectors \texttt{a} and \texttt{b}. By explicitly specifying 
that the input vectors are constant during the kernel execution (\ie \texttt{const}) and no other 
pointers are used to access them (\ie \texttt{restrict}), the GPU will cache the contents of that 
vectors, greatly reducing the kernel execution time.

\subsection{OpenCL Setup and Kernel Loading and Compilation}
Setting up the OpenCL run\hyp{}time tends to be a tedious and repetitive task which requires several 
lines of code. Analogously, loading the contents of files containing OpenCL kernels is also another 
annoyance of OpenCL programming. Listing~\ref{lst:opencl:init} shows the necessary source code to 
setup OpenCL, and load and compile the kernel code. GMAC requires no initialization code and 
provides a simplified path for kernel loading and compilation, as shown in 
Listing~\ref{lst:hpe:load}. 

\lstinputlisting[float,
    language=C,
    frame=tb,
    caption={GMAC\slash HPE code to make OpenCL code available to the application.},
    label={lst:hpe:load}]{hpe/load.c}

\subsection{Allocating and Releasing Data Structures}
Data structures used by kernels are allocated using the \texttt{eclMalloc(void **, size\_t)} API 
call. The first parameter is a pointer to the variable which will hold the CPU pointer used to 
access the allocated data structure. The second parameter is the size, in bytes, of the memory to be 
allocated. Finally, \texttt{eclMalloc()} returns an error code specifying whether the allocation 
succeeded (\texttt{eclSuccess}) or the condition that prevented the memory allocation.  
Listing~\ref{lst:hpe:alloc} shows the source code required to allocate the input and output vectors.  

\lstinputlisting[float,
    language=C,
    frame=tb,
    caption={GMAC\slash HPE code to allocate and initialized the input and output vectors.},
    label={lst:hpe:alloc}]{hpe/alloc.c}

Note that the CPU pointer returned by GMAC can be passed as a parameter to any other function (\eg 
\texttt{fread()} in Listing~\ref{lst:hpe:alloc}). This code also uses the GMAC API call 
\texttt{eclFree(void *)}, which is used to release the memory allocated by calls to 
\texttt{eclMalloc()}.

\subsection{Calling Kernels}
GMAC\slash HPE uses a kernel call interface similar to OpenCL\@. The programmer is required to set 
the kernel dimensions, pass the kernel parameters, and call the kernel. Listing~\ref{lst:hpe:call} 
shows the code required to invoke the vector addition kernel.

\lstinputlisting[float,
    language=C,
    frame=tb,
    caption={GMAC\slash HPE code to call the vector addition kernel.},
    label={lst:hpe:call}]{hpe/call.c}

First, a handler to the kernel to be called is obtained using \texttt{eclGetKernel()}. This handler 
is used to pass the parameters to the kernel (\texttt{eclSetKernelArg()} for values and 
\texttt{eclSetKernelArgPtr()} for pointers returned by \texttt{eclMalloc} and 
\texttt{eclGlobalMalloc}), and set the kernel dimensions and call the kernel 
(\texttt{eclKernelLaunch()}).

\subsection{Final Details}
After calling the vector addition kernel, the application writes the output vector to a file, and 
releases the allocated memory using \texttt{eclFree()}, as shown in Listing~\ref{lst:hpe:release}.

\lstinputlisting[float,
    language=C,
    frame=tb,
    caption={GMAC\slash HPE code to write the output vector and release memory.},
    label={lst:hpe:release}]{hpe/release.c}



% vim: set spell ft=tex fo=aw2t expandtab sw=2 tw=100:
