\section{Task Decomposition}
Task decomposition is an implementation technique where applications are divided in units of work 
(\ie tasks), which might have inter\hyp{}dependencies among them (\ie the output of a task is the 
input for another task). If none of the inputs of one task depends on the outputs of another task, 
these two tasks are independent and, therefore, can be executed in parallel. Hence, task 
decomposition is an implementation technique that provides an amenable way of exploiting parallel 
execution.

Software pipelining is a very common usage pattern of task decomposition in those applications where 
one or several tasks can independently process blocks of their input data. A quite popular example 
are video processing applications, where image frames can be processed one after the other.

\begin{figure}
\centering
\includegraphics[width=0.75\linewidth]{hpe/figures/task}
\caption{Task decomposition for the vector addition example.}
\label{fig:hpe:task}
\end{figure}

Figure~\ref{fig:hpe:task} shows a potential task decomposition for our vector addition example. The 
application is divided in three tasks: \emph{file read}, \emph{vector addition}, and \emph{file 
write}. In order to obtain performance improvements, the input vectors are divided in several 
blocks, being each input data block processed independently. Moreover, the \emph{file read} task is 
duplicated so each task will read one of the input vectors.

The most common practice to implement task\hyp{}based applications is to encapsulate each data 
structure and two semaphores in a single type, as illustrated in Listing~\ref{lst:hpe:structure} for 
the task\hyp{}based implementation of the vector addition example. The \texttt{ready} semaphore 
signals dependent tasks that the data structure is ready to be consumed. Each task executes as many 
\emph{post} operations over the \texttt{ready} semaphore as dependent tasks take the data structure 
as input. The \texttt{reuse} semaphore is only required if data structures are reused during the 
application execution, which is a common technique to avoid the overheads of memory allocation.  
Producer tasks perform as many \emph{wait} operations of the \texttt{reuse} semaphore as dependent 
tasks take the data structure as input parameter. Task consuming data structures perform a 
\emph{wait} operation over the \texttt{ready} semaphore before using each data structure, and a 
\emph{post} operation over the \texttt{reuse} semaphore when data structures are not longer used.

\lstinputlisting[float,
    language=C,
    frame=tb,
    caption={Data type used to encapsulate data structures in the task\hyp{}based vector addition 
    example.},
    label={lst:hpe:structure}]{hpe/structure.h}

The most simple implementation of task\hyp{}based application consists of a CPU control thread and 
as many worker CPU threads as task de application is decomposed to. Listing~\ref{lst:hpe:init} shows 
the initialization code for the CPU control thread in our vector addition example. The CPU control 
thread allocates the data structures used by the applications, initializing the \texttt{ready} and 
\texttt{reuse} semaphores for each of them to zero. Data structures in this application are 
allocated twice to allow double\hyp{}buffering of input and output task data structures and, thus, 
enabling parallel execution of tasks.

\lstinputlisting[float,
    language=C,
    frame=tb,
    caption={CPU control thread code to initialize data structures for a task\hyp{}based 
    implementation of vector addition using GMAC\slash HPE.},
    label={lst:hpe:init}]{hpe/init.c}

Then, the control CPU thread spawns one CPU thread for each task, and then waits for all of them to 
complete. Listing~\ref{lst:hpe:task} shows the source code for the control thread in the vector 
addition example, which follows the aforementioned structure. This code is exactly the very same 
code a task\hyp{}base application would implement if GMAC\slash HPE were not used.
\lstinputlisting[float,
    language=C,
    frame=tb,
    caption={CPU control thread in a task\hyp{}based implementation of vector addition.},
    label={lst:hpe:task}]{hpe/task.c}

Finally, the control code waits for each of the spawned threads to finish, and releases the data 
structures being used.

The code for worker threads implementing each task, consists of a loop where different blocks of the 
input and output vectors are processed. This code implements the synchronization mechanism 
previously discussed using \emph{post} and \emph{wait} operations over the semaphores of the data 
structures they take as input and output.

A final consideration is the case where a tasks execute kernels that take as input parameters data 
structures produced by other tasks. The GMAC memory model only allows kernels to access data 
structures allocated by the CPU thread performing the allocation. Hence, all these tasks should be 
executed by the same CPU thread. To overcome this difficulty, the application might use the
\texttt{ocl\-Device\-Copy()} and \texttt{ocl\-Device\-Receive()} GMAC calls. The former makes a copy 
of the virtual GPU associated to the calling CPU thread and sends it to a destination CPU thread.  
The latter discards the virtual GPU associated to the calling thread, and waits for another thread 
to send a virtual GPU\@. The combination of these two GMAC calls provides a simple mechanism for 
several CPU threads to shared the same virtual GPU.

Two additional GMAC calls related to virtual GPU management are \texttt{ocl\-Device\-Send()} and 
\texttt{ocl\-Device\-Send\-Receive()}. The former sends the virtual GPU associated to the calling 
thread to a destination thread, and, thus, the caller cannot uses its virtual GPU any more. If a new 
GMAC call is performed by the caller, a new virtual GPU will be created and associated to that CPU 
thread. The latter, \texttt{ocl\-Device\-Send\-Receive()} allows the caller CPU thread to sends its 
virtual GPU to a destination CPU thread, and waits for another CPU thread to send or copy its 
virtual GPU to the caller. 

% vim: set spell ft=tex fo=aw2t expandtab sw=2 tw=100:
