\section{The GMAC Memory Model}

The GMAC library builds an asymmetric distributed shared memory virtual space for systems formed by 
general purpose CPUs and one or several GPUs. Figure~\ref{fig:overview:memory} outlines this shared 
virtual memory model, where CPUs and GPUs access a common virtual address space. Specifically, each 
GPU cannot only access those memory locations that are hosted by its own memory.

\begin{figure}
\centering
\includegraphics[width=0.8\linewidth]{model/figures/memory}
\caption{Asymmetric Virtual Address Space in GMAC\@. CPUs can access any address within the virtual 
address space, whereas GPUs can only access a small portion of the virtual address space.}
\label{fig:overview:memory}
\end{figure}

A major consequence, and advantage, of the GMAC memory model is the lack of memory copy calls (\eg 
\texttt{clEnqueueReadBuffer()} / \texttt{clEnqueueWriteBuffer()} in applications source code. By 
removing explicit data transfers in the application code, GMAC eases the task of programming GPU 
systems. First, GMAC leverages programmers from the burden of tracking which processor (\ie CPU or 
GPU) has modified data structures last time, and coding the necessary calls to access data 
structures used by both CPUs and GPUs in a coherent way. Second, GMAC also avoids the extra coding 
necessary to only perform data transfers on those systems where CPUs and GPUs share the same 
physical memory such as in AMD Fusion APUs.



% vim: set spell ft=tex fo=aw2t expandtab sw=2 tw=100:
