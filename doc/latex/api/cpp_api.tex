\section{GMAC\slash HPE C++ types for OpenCL}

All symbols in the C++ API are nested in the \texttt{ocl} namespace.

\subsection{\texttt{ocl::error\_t}}

Most GMAC calls return an error code. \texttt{oclSuccess} is returned on success.
\vspace{11pt}\\
\begin{tabularx}{\linewidth}{|>{\texttt\bgroup}l<{\egroup}|X|}
  \hline
  \textnormal{Value} & Description \\
  \hline
  \hline
  oclSuccess                            & No error                             \\
  oclErrorMemoryAllocation              & Error allocating memory              \\
  oclErrorLaunchFailure                 & Error launching the kernel           \\
  oclErrorNotReady                      &                                      \\
  oclErrorNoAccelerator                 &                                      \\
  oclErrorInvalidValue                  & Invalid value passed to the function \\
  oclErrorInvalidAccelerator            & Invalid accelerator specified        \\
  oclErrorInvalidAcceleratorFunction    & Invalid accelerator function         \\
  oclErrorInvalidSize                   &                                      \\
  oclErrorAlreadyBound                  &                                      \\
  oclErrorApiFailureBase                &                                      \\
  oclErrorFeatureNotSupported           & Feature not supported by the hardware\slash{}compilation
                                          settings                             \\
  oclErrorInsufficientAcceleratorMemory & Not enough accelerator memory available to perform the 
                                          operation                            \\
  oclErrorInvalidKernelName             & Invalid kernel name                  \\
  oclErrorUnknown                       & Unknown error                        \\
  \hline
\end{tabularx}

\subsection{\texttt{ocl::kernel\_id\_t}}

Constant string with the name of a function to be executed on an accelerator.

\subsection{\texttt{ocl::memory\_hint}}

\begin{tabularx}{\linewidth}{|>{\texttt\bgroup}l<{\egroup}|X|}
  \hline
  \textnormal{Value} & Description \\
  \hline
  \hline
  OCL\_GLOBAL\_MALLOC\_CENTRALIZED & Prefer centralized (host\hyp{}mapped) implementation for global 
  malloc \\
  OCL\_GLOBAL\_MALLOC\_DISTRIBUTED & Prefer distributed copies implementation for global malloc \\
  \hline
\end{tabularx}


\subsection{\texttt{ocl::protection}}

\begin{tabularx}{\linewidth}{|>{\texttt\bgroup}l<{\egroup}|X|}
  \hline
  \textnormal{Value} & Description \\
  \hline
  \hline
  OCL\_PROT\_NONE  & No access rights for the mapping \\
  OCL\_PROT\_READ  & Read\hyp{}only access rights for the mapping \\
  OCL\_PROT\_WRITE & Write\hyp{}only access rights for the mapping \\
  OCL\_PROT\_READWRITE & Read\slash{}Write access rights for the mapping \\
  \hline
\end{tabularx}

\section{GMAC\slash HPE C++ API}

\subsection{\texttt{ocl::error\_t ocl::kernel::kernel()}}

\textbf{Description}: Creates a OpenCL kernel handler.

\subsection{\texttt{ocl::error\_t ocl::kernel::kernel(std::string name)}}

\textbf{Description}: Creates and gets an OpenCL kernel handler. \\
\textbf{Parameters}
\begin{itemize}
  \item \texttt{name}: string with the name of the OpenCL kernel to be retrieved
\end{itemize}

\subsection{\texttt{ocl::error\_t ocl::kernel::kernel(const char *name)}}

\textbf{Description}: Creates and gets an OpenCL kernel handler. \\
\textbf{Parameters}
\begin{itemize}
  \item \texttt{name}: Pointer to a NULL\hyp{}terminated string with the name of the OpenCL kernel 
to be retrieved
\end{itemize}

\subsection{\texttt{ocl::error\_t ocl::kernel::get(std::string name)}}

\textbf{Description}: Gets a OpenCL kernel handler. \\
\textbf{Parameters}
\begin{itemize}
  \item \texttt{name}: string with the name of the OpenCL kernel to be retrieved
\end{itemize}
\textbf{Returns} Error code

\subsection{\texttt{ocl::error\_t ocl::kernel::get(const char *name)}}

\textbf{Description}: Gets an OpenCL kernel handler. \\
\textbf{Parameters}
\begin{itemize}
  \item \texttt{name}: Pointer to a NULL\hyp{}terminated string with the name of the OpenCL kernel 
to be retrieved
\end{itemize}
\textbf{Returns} Error code

\subsection{\texttt{template <typename T>\\ocl::error\_t ocl::kernel::setArg(unsigned index, size\_t 
size, const T *addr)}}

\textbf{Description}: Adds an argument to be used by the following call to launch()\\
\textbf{Parameters}
\begin{itemize}
  \item \texttt{index}: Index of the parameter being added in the parameter list
  \item \texttt{size}: Size, in bytes, of the argument
  \item \texttt{addr}: Memory address where the param is stored
\end{itemize}
\textbf{Returns} Error code

\subsection{\texttt{template <typename T>\\ocl::error\_t ocl::kernel::setArg(unsigned index, T 
val)}}

\textbf{Description}: Adds an argument to be used by the following call to callNDRange()\\
\textbf{Parameters}
\begin{itemize}
  \item \texttt{index}: Index of the parameter being added to the parameter list
  \item \texttt{val}: Parameter to be stored
\end{itemize}
\textbf{Returns} Error code

\subsection{\texttt{ocl::error\_t callNDRange(size\_t workDim, size\_t *globalWorkOffset, size\_t 
*globalWorkSize, size\_t *localWorkSize)}}

\textbf{Description}: Launches a kernel execution with the specified device work configuration\\
\textbf{Parameters}
\begin{itemize}
  \item \texttt{workDim}: Number of elements for the work size arrays.
  \item \texttt{globalWorkOffset}: Array of \emph{workDim} unsigned elements that specifies the work 
  offset for the work items.
  \item \texttt{globalWorkSize}: Array of \emph{workDim} unsigned elements that specifies the global 
  number of work items.
  \item \texttt{localWorkSize}: Array of \emph{workDim} unsigned elements that specifies the number 
of work items per work group.
\end{itemize}
\textbf{Returns} Error code


\subsection{\texttt{ocl::error ocl::compileSource(const char *code, std::string flags = "")}}

\textbf{Description}: Prepares the OpenCL code to be used by the application. \\
\textbf{Parameters}
\begin{itemize}
  \item \texttt{code} String that contains the code to be prepared
  \item \texttt{flags}: String that contains the compilation flags or empty string
\end{itemize}
\textbf{Returns}: Error code

\subsection{\texttt{ocl::error ocl::compileSourceFile(std::string path, std::string flags = "")}}

\textbf{Description}: Prepares the OpenCL code in the specified file to be used by the application.  \\
\textbf{Parameters}
\begin{itemize}
  \item \texttt{path}: String that contains the path to the file with the source code to be prepared
  \item \texttt{flags}: String that contains the compilation flags or empty string
\end{itemize}
\textbf{Returns}: Error code

\subsection{\texttt{ocl::error ocl::compileSourceStream(std::istream \&in, std::string flags = "")}}

\textbf{Description}: Prepares the OpenCL code in the specified stream to be used by the 
application.  \\
\textbf{Parameters}
\begin{itemize}
  \item \texttt{path}: Reference to an input stream
  \item \texttt{flags}: Pointer to a NULL\hyp{}terminated string that contains the compilation flags or NULL
\end{itemize}
\textbf{Returns}: Error code

\subsection{\texttt{ocl::error ocl::compileBinary(const unsigned char *binary, size\_t size, 
std::string flags = "")}}

\textbf{Description}: Prepares the OpenCL binary to be used by the application.  \\
\textbf{Parameters}
\begin{itemize}
  \item \texttt{binary}: Pointer to the array that contains the binary code
  \item \texttt{size}: Size in bytes of the array that contains the binary code
  \item \texttt{flags}: String that contains the compilation flags or empty string
\end{itemize}
\textbf{Returns}: Error code

\subsection{\texttt{ocl::error ocl::compileBinaryFile(std::string path, std::string flags = "")}}

\textbf{Description}: Prepares the OpenCL code in the specified file to be used by the application.  \\
\textbf{Parameters}
\begin{itemize}
  \item \texttt{path}: String that contains the path to the file with the binary code to be prepared
  \item \texttt{flags}: String that contains the compilation flags or empty string
\end{itemize}
\textbf{Returns}: Error code

\subsection{\texttt{ocl::error ocl::compileBinaryStream(std::istream \&in, std::string flags = "")}}

\textbf{Description}: Prepares the OpenCL code in the specified stream to be used by the 
application.  \\
\textbf{Parameters}
\begin{itemize}
  \item \texttt{in}: Reference to an input stream that contains the binary code
  \item \texttt{flags}: String that contains the compilation flags or empty string
\end{itemize}
\textbf{Returns}: Error code


\subsection{\texttt{unsigned ocl::getNumberOfAccelerators()}}

\textbf{Description}: Get the number of available accelerators in the system
\textbf{Returns}: Number of accelerators

\subsection{\texttt{unsigned ocl::getFreeMemory()}}

\textbf{Description}: Gets the amount of available accelerator memory in bytes
\textbf{Returns}: Amount (in bytes) of the available accelerator memory

\subsection{\texttt{ocl::error ocl::migrate(unsigned acc)}}

\textbf{Description}: Attach the calling CPU thread to the given accelerator
\textbf{Parameters}
\begin{itemize}
  \item \texttt{acc}: Id of the accelerator to attach to
\end{itemize}
\textbf{Returns}: Error code

\subsection{\texttt{ocl::error ocl::malloc(void **devPtr, size\_t count)}}

\textbf{Description}: Allocate shared memory
\begin{itemize}
  \item \texttt{devPtr}: Memory address of the pointer to store the allocated memory
  \item \texttt{count}: Size (in bytes) of the memory to be allocated
\end{itemize}
\textbf{Returns}: Error code

\subsection{\texttt{ocl::error ocl::globalMalloc(void **devPtr, size\_t count, ocl\_memory\_hint 
hint dv(OCL\_GLOBAL\_MALLOC\_CENTRALIZED))}}

\textbf{Description}: Allocate shared memory accessible from all accelerators
\begin{itemize}
  \item \texttt{devPtr}: Memory address of the pointer to store the allocated memory
  \item \texttt{count}: Size (in bytes) of the memory to be allocated
  \item \texttt{hint}: Type of desired global memory
\end{itemize}
\textbf{Returns}: Error code

\subsection{\texttt{template <typename T>\\cl\_mem ocl::ptr(const T *cpuPtr)}}

\textbf{Description}: Get the OpenCL memory object associated to a shared memory address
\begin{itemize}
  \item \texttt{cpuPtr}: Host shared memory address
\end{itemize}
\textbf{Returns}: Associated OpenCL buffer

\subsection{\texttt{ocl::error ocl:free(const void *cpuPtr)}}

\textbf{Description}: Release shared memory
\begin{itemize}
  \item \texttt{cpuPtr}: Shared memory address to be released
\end{itemize}
\textbf{Returns}: Error code

\subsection{\texttt{ocl::error ocl::threadSynchronize()}}

\textbf{Description}: Wait until all previous accelerator calls are completed
\textbf{Returns}: Error code

\subsection{\texttt{ocl::error ocl::getLastError()}}

\textbf{Description}: Get the last error produced by GMAC
\textbf{Returns}: Error code

\subsection{\texttt{void *ocl::memset(void *cpuPtr, int c, size\_t count)}}

\textbf{Description}: Initialize a shared memory region
\begin{itemize}
  \item \texttt{cpuPtr}: Starting shared memory address.
  \item \texttt{c}: Value used to be initialized
  \item \texttt{count}: Size (in bytes) of the shared memory region to be initialized
\end{itemize}
\textbf{Returns}: Shared memory address that has been initialized

\subsection{\texttt{void *ocl::memcpy(void *cpuDstPtr, const void *cpuSrcPtr, size\_t count)}}

\textbf{Description}: Copy data between shared memory regions
\begin{itemize}
  \item \texttt{cpuDstPtr}: Destination shared memory
  \item \texttt{cpuSrcPtr}: Source shared memory
  \item \texttt{count}: Size (in bytes) to be copied
\end{itemize}
\textbf{Returns}: Destination shared memory address

\subsection{\texttt{void ocl::send(THREAD\_T tid)}}

\textbf{Description}: Send the execution mode associated to the current CPU thread to another CPU thread
\begin{itemize}
  \item \texttt{tid}: Thread ID of the destionation CPU thread
\end{itemize}

\subsection{\texttt{void ocl::receive()}}

\textbf{Description}: Receive an execution mode from another CPU thread

\subsection{\texttt{void ocl::sendReceive(THREAD\_T tid)}}

\textbf{Description}: Send the execution mode associated to the current CPU thread and wait to receive a new execution mode
\begin{itemize}
  \item \texttt{tid}: Thread ID of the destination CPU thread
\end{itemize}

\subsection{\texttt{void ocl::copy(THREAD\_T tid)}}

\textbf{Description}: Create a copy of the execution mode associate to the current CPU thread and send that copy another CPU thread
\begin{itemize}
  \item \texttt{tid}: Thread ID of the destination CPU thread
\end{itemize}

% vim: set spell ft=tex fo=aw2t expandtab sw=2 tw=100:
