\section{GMAC\slash HPE C types for OpenCL}

\subsection{\texttt{ecl\_error}}
Most GMAC calls return an error code. \texttt{eclSuccess} is returned on success.
\vspace{11pt}\\
\begin{tabularx}{\linewidth}{|>{\texttt\bgroup}l<{\egroup}|X|}
  \hline
  \textnormal{Value} & Description \\
  \hline
  \hline
  eclSuccess                            & No error                             \\
  eclErrorMemoryAllocation              & Error allocating memory              \\
  eclErrorLaunchFailure                 & Error launching the kernel           \\
  eclErrorNotReady                      &                                      \\
  eclErrorNoAccelerator                 &                                      \\
  eclErrorInvalidValue                  & Invalid value passed to the function \\
  eclErrorInvalidAccelerator            & Invalid accelerator specified        \\
  eclErrorInvalidAcceleratorFunction    & Invalid accelerator function         \\
  eclErrorInvalidSize                   &                                      \\
  eclErrorAlreadyBound                  &                                      \\
  eclErrorApiFailureBase                &                                      \\
  eclErrorFeatureNotSupported           & Feature not supported by the hardware\slash{}compilation
                                          settings                             \\
  eclErrorInsufficientAcceleratorMemory & Not enough accelerator memory available to perform the 
                                          operation                            \\
  eclErrorUnknown                       & Unknown error                        \\
  \hline
\end{tabularx}

\subsection{\texttt{ecl\_kernel\_id}}

Constant string with the name of a function to be executed on an accelerator.

\subsection{\texttt{ecl\_memory\_hint}}
\begin{tabularx}{\linewidth}{|>{\texttt\bgroup}l<{\egroup}|X|}
  \hline
  \textnormal{Value} & Description \\
  \hline
  \hline
  ECL\_GLOBAL\_MALLOC\_CENTRALIZED & Prefer centralized (host\hyp{}mapped) implementation for global 
  malloc \\
  ECL\_GLOBAL\_MALLOC\_DISTRIBUTED & Prefer distributed copies implementation for global malloc \\
  \hline
\end{tabularx}


\subsection{\texttt{ecl\_protection}}
\begin{tabularx}{\linewidth}{|>{\texttt\bgroup}l<{\egroup}|X|}
  \hline
  \textnormal{Value} & Description \\
  \hline
  \hline
  ECL\_PROT\_NONE  & No access rights for the mapping \\
  ECL\_PROT\_READ  & Read\hyp{}only access rights for the mapping \\
  ECL\_PROT\_WRITE & Write\hyp{}only access rights for the mapping \\
  ECL\_PROT\_READWRITE & Read\slash{}Write access rights for the mapping \\
  \hline
\end{tabularx}

\section{GMAC\slash HPE C API}

\subsection{\texttt{ecl\_error eclGetKernel(const char *name, ecl\_kernel *kernel)}}

\textbf{Description}: Gets a kernel handler for the speficied OpenCL kernel. \\
\textbf{Parameters}
\begin{itemize}
  \item \texttt{name}: Pointer to a NULL\hyp{}terminated string that contains the name of the kernel
  \item \texttt{kernel}: Pointer to store the address to the kernel handler
\end{itemize}
\textbf{Returns}: eclSuccess on success, an error code otherwise

\subsection{\texttt{ecl\_error eclSetKernelArg(ecl\_kernel kernel, unsigned index, size\_t size, 
const void *addr)}}

\textbf{Description}: Adds an argument to be used by the following call to eclCallNDRange() on 
the specified kernel \\
\textbf{Parameters}
\begin{itemize}
  \item \texttt{kernel}: Kernel handler
  \item \texttt{index}: Index of the parameter being added in the parameter list
  \item \texttt{size}: Size, in bytes, of the argument
  \item \texttt{addr}: Memory address where the param is stored
\end{itemize}
\textbf{Returns}: eclSuccess on success, an error code otherwise

\subsection{\texttt{ecl\_error eclCallNDRange(ecl\_kernel kernel, size\_t workDim, size\_t 
*globalWorkOffset, size\_t *globalWorkSize, size\_t *localWorkSize)}}

\textbf{Description}: Launches a kernel execution with the specified device work configuration\\
\textbf{Parameters}
\begin{itemize}
  \item \texttt{kernel}: Kernel handler of the kernel to be executed on the accelerator
  \item \texttt{workDim}: Number of elements for the work size arrays.
  \item \texttt{globalWorkOffset}: Array of \emph{workDim} unsigned elements that specifies the work 
    offset for the work items.
  \item \texttt{globalWorkSize}: Array of \emph{workDim} unsigned elements that specifies the global 
    number of work items.
  \item \texttt{localWorkSize}: Array of \emph{workDim} unsigned elements that specifies the number 
    of work items per work group.
\end{itemize}
\textbf{Returns}: eclSuccess on success, an error code otherwise

\subsection{\texttt{ecl\_error eclReleaseRelease(ecl\_kernel kernel)}}

\textbf{Description}: Releases the resources of the given kernel handler. \\
\textbf{Parameters}
\begin{itemize}
  \item \texttt{kernel}: Handler of the kernel to be released
\end{itemize}
\textbf{Returns}: eclSuccess on success, an error code otherwise


\subsection{\texttt{ecl\_error eclCompileSource(const char *code, const char *flags = NULL)}}

\textbf{Description}: Prepares the OpenCL code to be used by the application. \\
\textbf{Parameters}
\begin{itemize}
  \item \texttt{code} Pointer to the NULL\hyp{}terminated string that contains the code
  \item \texttt{flags} Compilation flags or NULL
\end{itemize}
\textbf{Returns}: eclSuccess on success, an error code otherwise

\subsection{\texttt{ecl\_error eclCompileSourceFile(const char *path, const char *flags = NULL)}}

\textbf{Description}: Prepares the OpenCL code in the specified file to be used by the application.  \\
\textbf{Parameters}
\begin{itemize}
  \item \texttt{path}: Pointer to a NULL\hyp{}terminated string pointing to the file with the code 
 to be prepared
  \item \texttt{flags}: \end{itemize}
\textbf{Returns}: eclSuccess on success, an error code otherwise

\subsection{\texttt{ecl\_error eclCompileBinary(const unsigned char *binary, size\_t size, const char *flags = NULL)}}

\textbf{Description}: Prepares the OpenCL binary to be used by the application.  \\
\textbf{Parameters}
\begin{itemize}
  \item \texttt{binary}: Pointer to the array that contains the binary code
  \item \texttt{size}: Size in bytes of the array that contains the binary code
  \item \texttt{flags}: Pointer to a NULL\hyp{}terminated string with the compilation flags or 
NULL
\end{itemize}
\textbf{Returns}: eclSuccess on success, an error code otherwise

\subsection{\texttt{ecl\_error eclCompileBinaryFile(const unsigned char *binary, const char 
*flags = NULL)}}

\textbf{Description}: Prepares the OpenCL binary in the specified file to be used by the application.  \\
\textbf{Parameters}
\begin{itemize}
  \item \texttt{path}: Pointer to a NULL\hyp{}terminated string pointing to the file with the binary 
  code to be prepared
  \item \texttt{flags}: Pointer to a NULL\hyp{}terminated string with the compilation flags or NULL
\end{itemize}
\textbf{Returns}: eclSuccess on success, an error code otherwise

\subsection{\texttt{unsigned eclGetNumberOfAccelerators()}}

\textbf{Description}: Get the number of available accelerators in the system
\textbf{Returns}: Number of accelerators

\subsection{\texttt{unsigned eclGetFreeMemory()}}

\textbf{Description}: Gets the amount of available accelerator memory in bytes
\textbf{Returns}: Amount (in bytes) of the available accelerator memory

\subsection{\texttt{ecl\_error eclMigrate(unsigned acc)}}

\textbf{Description}: Attach the calling CPU thread to the given accelerator
\textbf{Parameters}
\begin{itemize}
  \item \texttt{acc}: Id of the accelerator to attach to
\end{itemize}
\textbf{Returns}: eclSuccess on success, an error code otherwise

\subsection{\texttt{ecl\_error eclMalloc(void **devPtr, size\_t count)}}

\textbf{Description}: Allocate shared memory
\begin{itemize}
  \item \texttt{devPtr}: Memory address of the pointer to store the allocated memory
  \item \texttt{count}: Size (in bytes) of the memory to be allocated
\end{itemize}
\textbf{Returns}: eclSuccess on success, an error code otherwise

\subsection{\texttt{eclGlobalMalloc(void **devPtr, size\_t count, ecl\_memory\_hint hint dv(ECL\_GLOBAL\_MALLOC\_CENTRALIZED))}}

\textbf{Description}: Allocate shared memory accessible from all accelerators
\begin{itemize}
  \item \texttt{devPtr}: Memory address of the pointer to store the allocated memory
  \item \texttt{count}: Size (in bytes) of the memory to be allocated
  \item \texttt{hint}: Type of desired global memory
\end{itemize}
\textbf{Returns}: eclSuccess on success, an error code otherwise

\subsection{\texttt{cl\_mem eclPtr(const void *cpuPtr)}}

\textbf{Description}: Get the OpenCL memory object associated to a shared memory address
\begin{itemize}
  \item \texttt{cpuPtr}: Host shared memory address
\end{itemize}
\textbf{Returns}: Associated OpenCL buffer

\subsection{\texttt{eclFree(const void *cpuPtr)}}

\textbf{Description}: Release shared memory
\begin{itemize}
  \item \texttt{cpuPtr}: Shared memory address to be released
\end{itemize}
\textbf{Returns}: eclSuccess on success, an error code otherwise

\subsection{\texttt{eclThreadSynchronize()}}

\textbf{Description}: Wait until all previous accelerator calls are completed
\textbf{Returns}: eclSuccess on success, an error code otherwise

\subsection{\texttt{eclGetLastError()}}

\textbf{Description}: Get the last error produced by GMAC
\textbf{Returns}: eclSuccess on success, an error code otherwise

\subsection{\texttt{void *eclMemset(void *cpuPtr, int c, size\_t count)}}

\textbf{Description}: Initialize a shared memory region
\begin{itemize}
  \item \texttt{cpuPtr}: Starting shared memory address.
  \item \texttt{c}: Value used to be initialized
  \item \texttt{count}: Size (in bytes) of the shared memory region to be initialized
\end{itemize}
\textbf{Returns}: Shared memory address that has been initialized

\subsection{\texttt{void *eclMemcpy(void *cpuDstPtr, const void *cpuSrcPtr, size\_t count)}}

\textbf{Description}: Copy data between shared memory regions
\begin{itemize}
  \item \texttt{cpuDstPtr}: Destination shared memory
  \item \texttt{cpuSrcPtr}: Source shared memory
  \item \texttt{count}: Size (in bytes) to be copied
\end{itemize}
\textbf{Returns}: Destination shared memory address

%\subsection{\texttt{void eclSend(THREAD\_T tid)}}
%
%\textbf{Description}: Send the execution mode associated to the current CPU thread to another CPU thread
%\begin{itemize}
%  \item \texttt{tid}: Thread ID of the destionation CPU thread
%\end{itemize}
%
%\subsection{\texttt{void eclReceive()}}
%
%\textbf{Description}: Receive an execution mode from another CPU thread */
%
%\subsection{\texttt{void eclSendReceive(THREAD\_T tid)}}
%
%\textbf{Description}: Send the execution mode associated to the current CPU thread and wait to receive a new execution mode
%\begin{itemize}
%  \item \texttt{tid}: Thread ID of the destination CPU thread
%\end{itemize}
%
%\subsection{\texttt{void eclCopy(THREAD\_T tid)}}
%
%\textbf{Description}: Create a copy of the execution mode associate to the current CPU thread and send that copy another CPU thread
%\begin{itemize}
%  \item \texttt{tid}: Thread ID of the destination CPU thread
%\end{itemize}

% vim: set spell ft=tex fo=aw2t expandtab sw=2 tw=100:
