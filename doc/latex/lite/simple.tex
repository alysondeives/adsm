\section{Single\hyp{}Threaded Applications}

\subsection{Initialization}
GMAC\slash Lite requires data allocations to be performed over single\hyp{}device OpenCL contexts.  
This simple restriction allows the initialization of GMAC\slash Lite applications to be greatly 
simplified compared to regular OpenCL applications (Listing~\ref{lst:opencl:init}). GMAC\slash Lite 
offers a convenient API call (\texttt{oclLiteInit()}). This API call can be also combined with 
\texttt{oclLiteProgram()} to further simplify the load and compilation of OpenCL kernels from 
external files. The combination of these both GMAC\slash Lite API calls can be combined as 
illustrated in Listing~\ref{lst:lite:load}.

\lstinputlisting[float,
    language=C,
    frame=tb,
    caption={GMAC\slash Lite code to make OpenCL code available to the application.},
    label={lst:lite:load}]{lite/load.c}


\subsection{Memory Allocation}
Memory allocation in GMAC\slash Lite is done using the \texttt{clMalloc()} API call, which returns a 
CPU pointer that can be used in any place of the CPU code. Listing~\ref{lst:lite:load} shows the 
GMAC\slash Lite code to allocate data structures using \texttt{clMalloc()}. In this example we use 
the same \texttt{load\_vector()} function used in Listing~\ref{lst:hpe:load}.

\lstinputlisting[float,
    language=C,
    frame=tb,
    caption={GMAC\slash Lite code to allocate and initialized the input and output vectors.},
    label={lst:lite:alloc}]{lite/alloc.c}

\subsection{Kernel Calls}
Kernel calls in GMAC\slash Lite is very similar to both GMAC\slash HPE and OpenCL\@.  
Listing~\ref{lst:lite:call} shows the code to perform the kernel call, which uses 
\texttt{clBuffer()} to get the \texttt{cl\_mem} object associated to memory allocated through calls 
to \texttt{clMalloc()}.

\lstinputlisting[float,
    language=C,
    frame=tb,
    caption={GMAC\slash Lite code to call the vector addition kernel.},
    label={lst:lite:call}]{lite/call.c}

\subsection{Memory Release}
Listing~\ref{lst:lite:release} shows the source code to release resources in GMAC\slash Lite. Memory 
is released calling \texttt{clFree()}, while all data structures associated with OpenCL are released 
calling to \texttt{clLiteRelease()}.

\lstinputlisting[float,
    language=C,
    frame=tb,
    caption={GMAC\slash Lite code to write the output vector and release memory.},
    label={lst:lite:release}]{lite/release.c}

% vim: set spell ft=tex fo=aw2t expandtab sw=2 tw=100:
